\documentclass{sigchi}

% Use this command to override the default ACM copyright statement
% (e.g. for preprints).  Consult the conference website for the
% camera-ready copyright statement.

%% EXAMPLE BEGIN -- HOW TO OVERRIDE THE DEFAULT COPYRIGHT STRIP -- (July 22, 2013 - Paul Baumann)
% \toappear{Permission to make digital or hard copies of all or part of this work for personal or classroom use is      granted without fee provided that copies are not made or distributed for profit or commercial advantage and that copies bear this notice and the full citation on the first page. Copyrights for components of this work owned by others than ACM must be honored. Abstracting with credit is permitted. To copy otherwise, or republish, to post on servers or to redistribute to lists, requires prior specific permission and/or a fee. Request permissions from permissions@acm.org. \\
% {\emph{CHI'14}}, April 26--May 1, 2014, Toronto, Canada. \\
% Copyright \copyright~2014 ACM ISBN/14/04...\$15.00. \\
% DOI string from ACM form confirmation}
%% EXAMPLE END -- HOW TO OVERRIDE THE DEFAULT COPYRIGHT STRIP -- (July 22, 2013 - Paul Baumann)

% Arabic page numbers for submission.  Remove this line to eliminate
% page numbers for the camera ready copy
% \pagenumbering{arabic}

% Load basic packages
\usepackage{balance}  % to better equalize the last page
\usepackage{graphics} % for EPS, load graphicx instead 
\usepackage[T1]{fontenc}
\usepackage{txfonts}
\usepackage{mathptmx}
\usepackage[pdftex]{hyperref}
\usepackage{color}
\usepackage{booktabs}
\usepackage{textcomp}
% Some optional stuff you might like/need.
\usepackage{microtype} % Improved Tracking and Kerning
% \usepackage[all]{hypcap}  % Fixes bug in hyperref caption linking
\usepackage{ccicons}  % Cite your images correctly!
% \usepackage[utf8]{inputenc} % for a UTF8 editor only

% If you want to use todo notes, marginpars etc. during creation of your draft document, you
% have to enable the "chi_draft" option for the document class. To do this, change the very first
% line to: "\documentclass[chi_draft]{sigchi}". You can then place todo notes by using the "\todo{...}"
% command. Make sure to disable the draft option again before submitting your final document.
\usepackage{todonotes}

% Paper metadata (use plain text, for PDF inclusion and later
% re-using, if desired).  Use \emtpyauthor when submitting for review
% so you remain anonymous.
\def\plaintitle{Visual Estimation of Trend in Scatterplot Visualizations}
\def\plainauthor{Removed for Review}
\def\emptyauthor{}
\def\plainkeywords{Information Visualization, Graphical Perception, Regression}
\def\plaingeneralterms{Visualization}

% llt: Define a global style for URLs, rather that the default one
\makeatletter
\def\url@leostyle{%
  \@ifundefined{selectfont}{
    \def\UrlFont{\sf}
  }{
    \def\UrlFont{\small\bf\ttfamily}
  }}
\makeatother
\urlstyle{leo}

% To make various LaTeX processors do the right thing with page size.
\def\pprw{8.5in}
\def\pprh{11in}
\special{papersize=\pprw,\pprh}
\setlength{\paperwidth}{\pprw}
\setlength{\paperheight}{\pprh}
\setlength{\pdfpagewidth}{\pprw}
\setlength{\pdfpageheight}{\pprh}

% Make sure hyperref comes last of your loaded packages, to give it a
% fighting chance of not being over-written, since its job is to
% redefine many LaTeX commands.
\definecolor{linkColor}{RGB}{6,125,233}
\hypersetup{%
  pdftitle={\plaintitle},
% Use \plainauthor for final version.
%  pdfauthor={\plainauthor},
  pdfauthor={\emptyauthor},
  pdfkeywords={\plainkeywords},
  bookmarksnumbered,
  pdfstartview={FitH},
  colorlinks,
  citecolor=black,
  filecolor=black,
  linkcolor=black,
  urlcolor=linkColor,
  breaklinks=true,
}

% create a shortcut to typeset table headings
% \newcommand\tabhead[1]{\small\textbf{#1}}

% End of preamble. Here it comes the document.
\begin{document}

\title{\plaintitle}

\numberofauthors{2}
\author{%
%  \alignauthor{Michael Correll\\
%    \affaddr{University of Washington}\\
%    \email{mcorrell@cs.washington.edu}}\\
%  \alignauthor{Jeffrey Heer\\
%    \affaddr{University of Washington}\\
%    \email{jheer@cs.washington.edu}}\\
}

\maketitle

\begin{abstract}
Observing trends in bivariate data, or predicting the future values of data, is a common task for viewers of visualizations. Yet, designer do not always explicitly include trend lines and other relevant statistical modeling information in charts. Thus, viewers must often perform \emph{regression by eye} --- they estimate the trend of data through the values alone. If viewer performance at this task is sufficiently poor, or if certain designs introduce unacceptable bias, then designers will need to rethink how to communicate trends to the general audience. In this work, we present a series of crowd-sourced experiments examining regression by eye, investigating both viewer performance at estimation of trends in bivariate data, and potential sources of bias in these estimations. Our findings indicate that, while viewers can accurately estimate trends in many standard visualizations of bivariate data, certain visual designs (such as the visual asymmetry of area charts) and certain features of the data (such as the presence of outliers) result in visual estimations that do not align with basic statistical models of regression.
\end{abstract}

\category{H.5.m.}{Information Interfaces and Presentation
  (e.g. HCI)}{Miscellaneous}

\keywords{\plainkeywords}

\section{Introduction}



\section{Related Work}

\section{General Methods}

In order to assess the ability of viewers of visualizations to perform regression-by-eye, we conducted a series of three crowd-sourced experiments on Amazon's Mechanical Turk platform. We limited the participant pool to subjects from within the United States, with a prior task approval rating of at least 90\%. For all experiments, we included stimuli where the trend line was visible as attention checks and for validation. We excluded participants with poor performance on these stimuli. Based on timings from internal piloting, we paid each participant \$2 for their participation, for an intended rate of \$8/hour. 

We generated the scatterplots by first generating the horizontal position of points through banded Monte Carlo sampling, ensuring that, while the distribution of points in X appeared uniform, at least X/b points appear in each of the b bands. We performed rejection sampling to ensure that no two points overlapped by more than half their radius in the resulting 600x1050 image. Next, we created a set of residual values, sampled evenly from discretized gaussian. This procedure ensured that the actual and intended bandwidth and standard deviation of residual values is precise. This set was permuted, and then applied to the points along with a target linear trend. Heteroskedasticity introduced through permutation could alter the actual linear trend away from the target trend, and so, we performed rejection sampling to ensure that the trend of the resulting points were within $10^{-7}$ of the target trend. We reused this residuals across all different trend types (linear, quadratic, or trigonometric). Except where noted, we selected trend lines that were centered in the image: that is, for a vertical and horizontal data extent of $[0,1]$, $f(0.5) = 0.5$.

\subsection{Participants}


\section{Estimation of Slope is Reliable \\ Across Visualizations}

We presented participants with a series of scatterplots, and adjusted a slider to fit the perceived trend in the points. Scatterplots consisted of one of three types of trend (linear, quadratic, or trigonometric). For each scatterplot, participants adjusted a slider that controlled the slope of a rendered trend line (or, in the case of the quadratic trends, the curvature, or the trigonometric, the positive/negative amplitude).


\subsection{Hypotheses}
\subsection{Results}
\subsection{Discussion}

\section{Estimation of Intercept is Subject to \\ ``Within The Bar'' Bias}
\subsection{Hypotheses}
\subsection{Results}
\subsection{Discussion}

\section{Estimation of Slope is Sensitive to Outliers}
\subsection{Hypotheses}
\subsection{Results}
\subsection{Discussion}

\section{Discussion}
\subsection{Limitations \& Future Work}
\subsection{Conclusion}
\section{Acknowledgments}

Omitted for review.

% Balancing columns in a ref list is a bit of a pain because you
% either use a hack like flushend or balance, or manually insert
% a column break.  http://www.tex.ac.uk/cgi-bin/texfaq2html?label=balance
% multicols doesn't work because we're already in two-column mode,
% and flushend isn't awesome, so I choose balance.  See this
% for more info: http://cs.brown.edu/system/software/latex/doc/balance.pdf
%
% Note that in a perfect world balance wants to be in the first
% column of the last page.
%
% If balance doesn't work for you, you can remove that and
% hard-code a column break into the bbl file right before you
% submit:
%
% http://stackoverflow.com/questions/2149854/how-to-manually-equalize-columns-
% in-an-ieee-paper-if-using-bibtex
%
% Or, just remove \balance and give up on balancing the last page.
%
\balance{}


% REFERENCES FORMAT
% References must be the same font size as other body text.
\bibliographystyle{SIGCHI-Reference-Format}
\bibliography{sample}

\end{document}

%%% Local Variables:
%%% mode: latex
%%% TeX-master: t
%%% End:
